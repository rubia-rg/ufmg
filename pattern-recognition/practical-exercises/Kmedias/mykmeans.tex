\documentclass{article}

\usepackage[utf8]{inputenc}
\usepackage[brazil]{babel}

\title{Exercício 3: K-means Clustering}
\author{Rúbia Reis Guerra \\ 2013031143}

\usepackage{Sweave}
\begin{document}
\Sconcordance{concordance:mykmeans.tex:mykmeans.Rnw:%
1 8 1 1 0 5 1 1 2 1 0 1 1 1 7 6 0 1 22 23 0 1 2 1 3 2 0 1 1 1 25 27 0 1 %
2 3 1}


\section{K-means}

\subsection{Funções}
\begin{Schunk}
\begin{Sinput}
> rm(list=ls())
> library('MASS')
> distance <- function(xt, centers){
+   distMatrix <- matrix(NA, nrow=dim(xt)[1], ncol=dim(centers)[1])
+   for(i in 1:nrow(centers)) {
+     distMatrix[,i] <- sqrt(rowSums(t(t(xt)-centers[i,])^2))
+   }
+   distMatrix
+ }
> mykmeans <- function(x, k, maxIter) {
+   clusterOld <- c()
+   centerOld <- c()
+   centers <- x[sample(nrow(x), k),]
+   
+   flag <- FALSE
+   i <- 0
+   while(i <= maxIter && flag==FALSE) {
+     i <- i + 1
+     if(i > 1) {
+       clusterOld <- clusters
+       centerOld <- centers
+     }
+     distsToCenters <- distance(x, centers)
+     clusters <- apply(distsToCenters, 1, which.min)
+     centers <- apply(x, 2, tapply, clusters, mean)
+     flag <- identical(clusters,clusterOld)
+   }
+ 
+   list(clusters=clusters, centers=centers)
+ }
\end{Sinput}
\end{Schunk}

\begin{Schunk}
\begin{Sinput}
> ##################################################################
> k <- c(2, 4, 8)
> sd2 <- (c(0.3, 0.5, 0.7)^2)
> for(j in 1:length(sd))
+ {
+   for(i in 1:length(k))
+   { 
+     N <- 100
+     maxIter <- 100
+     cores <- rainbow(k[i])
+     
+     S <- matrix(c(sd2[j],0,0,sd2[j]),byrow=T,ncol=2)
+     g1 <- mvrnorm(N,mu=c(2,2), Sigma=S)
+     g2 <- mvrnorm(N,mu=c(2,4), Sigma=S)
+     g3 <- mvrnorm(N,mu=c(4,2), Sigma=S)
+     g4 <- mvrnorm(N,mu=c(4,4), Sigma=S)
+     samples <- rbind(g1,g2,g3,g4)
+     
+     b <- mykmeans(samples, k[i], maxIter)
+     for(l in 1:(k[i]-1))
+     {
+       plot(samples[b$clusters==l,1],samples[b$clusters==l,2],type='p',col=cores[l],xlab='',ylab='',xlim=c(0,6),ylim=c(0,6))
+       par(new=T)
+     }
+     
+     plot(samples[b$clusters==k[i],1],samples[b$clusters==k[i],2],type='p',col=cores[k[i]],xlab='',ylab='',xlim=c(0,6),ylim=c(0,6))
+   }
+ }
\end{Sinput}
\end{Schunk}



\end{document}
