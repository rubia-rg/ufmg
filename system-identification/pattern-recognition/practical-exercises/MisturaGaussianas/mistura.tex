\documentclass{article}

\usepackage[utf8]{inputenc}
\usepackage[brazil]{babel}

\title{Exercício 4: Mistura de Gaussianas}
\author{Rúbia Reis Guerra \\ 2013031143}

\usepackage{Sweave}
\begin{document}
\Sconcordance{concordance:mistura.tex:mistura.Rnw:%
1 8 1 1 0 7 1 1 2 1 0 3 1 3 0 1 2 1 4 3 0 3 1 1 4 2 0 8 1 1 56 54 0 1 2 %
5 0 1 1 6 0 1 2 2 1}

\maketitle

\section{Classificador Bayesiano: Mistura de Gaussianas}
Nesta atividade, foi proposta a amostragem de dados do dataset BreastCancer, seguida da divisão em conjuntos de teste e treino e classifiação bayesiana. Para obter as densidades de cada classe, foi utilizado o pacote \textit{mclust}.

\subsection{Pacotes utilizados}
\begin{Schunk}
\begin{Sinput}
> rm(list=ls())
> library('MASS')
> library('mlbench')
> library('mclust')
\end{Sinput}
\end{Schunk}
\subsection{Implementação}
\begin{Schunk}
\begin{Sinput}
> ###########################
> # Dataset BreastCancer #
> data(BreastCancer)
> X<- data.matrix(BreastCancer[,2:10])
> X[is.na(X)] <- 0
> Y <- as.numeric(BreastCancer$Class)
> ###########################
> # Auxiliares #
> tp <- c()
> fp <- c()
> fn <- c()
> prec <- c()
> rec <- c()
> f1 <- c()
> error <- c()
> mse <- c()
> sde <- c()
> for(j in 1:10){
+   ###########################
+   # Amostrar dados #
+   index <- sample(2, nrow(BreastCancer), replace=TRUE, prob=c(0.70,0.30))
+   
+   ###########################
+   # Conjunto de treinamento #
+   training <- X[which(index==1),]
+   trainingLabels <- as.matrix(Y[which(index==1)])
+   
+   ###########################
+   # Conjunto de teste #
+   test <- X[which(index==2),]
+   testLabels <- as.matrix(Y[which(index==2)])
+   
+   ###########################
+   # Probabilidades a priori #
+   pc1 <- length(Y[which(Y==1)])/(length(Y))
+   pc2 <- length(Y[which(Y==2)])/(length(Y))
+   
+   ###########################
+   # Treinamento #
+   mod1 = densityMclust(training[which(trainingLabels==1),])
+   mod2 = densityMclust(training[which(trainingLabels==2),])
+   
+   ###########################
+   # Teste #
+   pxc1 <- dens(modelName=mod1$modelName, data = test, parameters = mod1$parameters)
+   pxc2 <- dens(modelName=mod2$modelName, data = test, parameters = mod2$parameters)
+   
+   ###########################
+   # Classificação #
+   Ntest <- dim(test)[1]
+   testY <- c()
+   for(i in 1:Ntest)
+   {
+     testY[i] <- ifelse(pxc1[i]/pxc2[i] >= pc2/pc1, 1, 2)
+     error[i] <- (testY[i]-testLabels[i])^2
+   }
+   
+   # MSE e SD #
+   mse[j] <- mean(error)
+   sde[j] <- sd(error)
+   
+   # Matriz de confusao #
+   testCM <- table(testY,testLabels)
+   
+   # Precision, recall, F1 #
+   tp[j] <- sum((testY==1) & (testLabels==1)) # True positives
+   fp[j] <- sum((testY==1) & (testLabels==2)) # False positives
+   fn[j] <- sum((testY==2) & (testLabels==1)) # False negatives
+   prec[j] <- tp[j]/(tp[j] + fp[j]) # Precision
+   rec[j] <- tp[j]/(tp[j] + fn[j]) # Recall
+   f1[j] <- 2*prec[j]*rec[j]/(prec[j]+rec[j]) # F1 Score
+ }
> mean(mse) # MSE
\end{Sinput}
\begin{Soutput}
[1] NA
\end{Soutput}
\begin{Sinput}
> mean(sde) # SD
\end{Sinput}
\begin{Soutput}
[1] NA
\end{Soutput}
\end{Schunk}
Por alguma razão ainda não identificada, a utilização de \textit{densityMclust} para os dados da classe 2 resulta sempre em densidades nulas, que, por sua vez, força o classificador ter saída 1 para todos os dados amostrados.

\end{document}
